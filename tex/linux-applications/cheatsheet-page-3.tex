% copyright (c): 2019 Tony Aldon <aldon.tony@gmail.com>
% License: MIT

\begin{picture}(297,210)
  \put(10,200){
		\begin{minipage}[t]{85mm}
      \section{keyboard shortcuts}{(3/4)}{-for linux applications-} \

      \begin{fctenv} 
        
        new tab |
        restore tab |
        duplicate tab 
      \end{fctenv}

      \sepwithinsubpar
      
      \customkeybinding{\klistA{r}{m}{s p}}

      \begin{fctenv} 

        refresh tab |
        toggle mute tab |
        toggle pin tab 
      \end{fctenv}
      
      \sepwithinsubpar

      \customkeybinding{\klistA{x}{c o}{c u}{c i}}

      \begin{fctenv} 

        close tab |
        close other tabs |
        close left tabs |
        close right tabs 
      \end{fctenv}

      \sepwithinsubpar

      \customkeybinding{\klistA{a}{u}{i}{e}}

      \begin{fctenv} 
        
        first tab |
        previous tab |
        next tab |
        last tab 
      \end{fctenv}

      \sepwithinsubpar

      \customkeybinding{\klistA{b}{f}{s a}{s e}}

      \begin{fctenv} 

        move tab left |
        move tab right |
        move tab first |
        move tab last 
      \end{fctenv}

      \subparagraph{windows}

      \customkeybinding{\klistA{o o}{s o}}

      \begin{fctenv} 

        new window |
        move tab new window 
      \end{fctenv}
      
    \end{minipage}
	}

  \put(105,200){
		\begin{minipage}[t]{85mm}

      \paragraph{i3 window management}
      
      {\footnotesize
        When using i3 program, some gnome internal programs are not awailable or
        reduced to basics stuff. For example, you have to set your keyboard
        layout by command line interface.

        I use the modular key 'window key' (W) to define some of the key bindings.

        To reload or restart i3, you can run in the terminal 'i3-msg reload'
        or 'i3-msg restart'.
      }

      \paragraphpart{direct actions}

      \klistA{<f10>}

      \begin{fctenv} 
        
        fullscreen toggle
      \end{fctenv}
      
      \subparagraph{dmenu/kill/exit}

      \klistA{W-m}{W-q}{W-S-q}

      \begin{fctenv} 
        
        dmenu |
        kill focused window |
        exit i3
      \end{fctenv}

      \subparagraph{start applications}

      \klistA{W-1}{W-2}{W-3}
      
      \begin{fctenv} 
        
        emacs |
        i3-sensible-terminal |
        chromium-browser
      \end{fctenv}

      \subparagraph{split}

      \klistA{W-h}{W-m}

      \begin{fctenv} 
        
        horizontal split |
        vertical split
      \end{fctenv}

      \subparagraph{focus}

      \klistA{W-}\sepmodekeyAkeyB\keybox{\klistB{s}{r}{d}{l}}

      \begin{fctenv} 

        focus left |
        focus right |
        focus up |
        focus down
      \end{fctenv}
      
      
    \end{minipage}
  }


  \put(200,200){
		\begin{minipage}[t]{85mm}

      \klistA{W-S-}\sepmodekeyAkeyB\keybox{\klistB{s}{r}{d}{l}}

      \begin{fctenv} 

        move left |
        move right |
        move up |
        move down
      \end{fctenv}

      \subparagraph{workspaces}

      \klistA{C-<f3>}{C-<f4>}{C-<f5>}

      \begin{fctenv} 

        switch to workspace 1 |
        switch to workspace 2 |
        switch to workspace 3
      \end{fctenv}

      \sepwithinsubpar
      
      \klistA{W-S-1}{W-S-2}{W-S-3}

      \begin{fctenv} 

        move workspace 1 |
        move workspace 2 |
        move workspace 3
      \end{fctenv}


      \paragraphpart{mode layout}

      \klistA{<f3>}

      \begin{fctenv}
        
        enter in mode layout
      \end{fctenv}

      \sepwithinsubpar

      \emacsmode{layout}\sepmodekeyAkeyB\klistA{<return>}

      \begin{fctenv}
        
        quit mode layout
      \end{fctenv}


      \sepwithinsubpar

      \emacsmode{layout}\sepmodekeyAkeyB\keybox{\klistB{u}{i}{e}}


      
      \begin{fctenv}
        
        layout stacking |
        layout tabbed |
        layout toggle split
      \end{fctenv}

      \sepwithinsubpar

      \emacsmode{layout}\sepmodekeyAkeyB\keybox{\klistB{p}{n}}

      \begin{fctenv}
        
        focus parent |
        focus child
      \end{fctenv}

    \end{minipage}
  }

  \contact
  
\end{picture}